%\documentclass[12pt,a4paper]{amsart}
\documentclass[10pt,a4paper]{article}
\ifx\pdfpageheight\undefined\PassOptionsToPackage{dvips}{graphicx}\else%
\PassOptionsToPackage{pdftex}{graphicx}
\PassOptionsToPackage{pdftex}{color}
\fi

\usepackage{diagrams1}
\usepackage[all]{xy}
\usepackage{url}
\usepackage{verbatim}
\usepackage{latexsym}
\usepackage{amssymb,amstext,amsmath}
\usepackage{epsf}
\usepackage{epsfig}
% \usepackage{isolatin1}
\usepackage{a4wide}
\usepackage{verbatim}
\usepackage{proof}
\usepackage{latexsym}
\usepackage{amssymb}
% \usepackage{stmaryrd}
\newcommand{\abs}[2]{\lambda #1 . #2}            % abstraction of #1 in #2
\usepackage{mytheorems}


\usepackage{float}
\floatstyle{boxed}
\restylefloat{figure}


%%%%%%%%%
\def\oge{\leavevmode\raise
.3ex\hbox{$\scriptscriptstyle\langle\!\langle\,$}}
\def\feg{\leavevmode\raise
.3ex\hbox{$\scriptscriptstyle\,\rangle\!\rangle$}}

%%%%%%%%%


\newcommand\myfrac[2]{
 \begin{array}{c}
 #1 \\
 \hline \hline 
 #2
\end{array}}


\newcommand*{\Scale}[2][4]{\scalebox{#1}{$#2$}}%
\newcommand*{\Resize}[2]{\resizebox{#1}{!}{$#2$}}

%\newcommand{\lam}[2]{{\langle}#1{\rangle}#2}
\newcommand{\north}{{\sf N}}
\newcommand{\south}{{\sf S}}
\newcommand{\merid}{{\sf merid}}
\newcommand{\elim}{{\sf elim}}
\newcommand{\openb}[1]{\mathsf{b}(#1)}
\newcommand{\dM}{{\sf dM}}
\newcommand{\Cir}{\mathsf{S^1}}
\newcommand{\inh}{\mathsf{inh}}
\newcommand{\squash}{{\sf squash}}
\newcommand{\transp}{{\sf transp}}
\newcommand{\squeeze}{{\sf squeeze}}
\newcommand{\lop}{\mathsf{loop}}
\newcommand{\coerce}{\mathsf{comp}}
\newcommand{\inc}{\mathsf{inc}}
\newcommand{\base}{\mathsf{base}}
\newcommand{\Top}{{\sf Top}}
\newcommand{\Sys}{{\sf S}}
\newcommand{\Ref}{{\sf Ref}}
\newcommand{\LINE}{{\sf line}}
\newcommand{\JJ}{{\sf J}}
\newcommand{\II}{\mathbb{I}}
\newcommand{\PP}{{\sf Path}}
\newcommand{\pp}{\mathsf{p}}
\newcommand{\Sp}{{\sf S}}
\newcommand{\sprop}{\mathsf{sprop}}
\newcommand{\sset}{\mathsf{sset}}
\newcommand{\bset}{\mathsf{bset}}
\newcommand{\sProp}{\mathsf{sProp}}
\newcommand{\sSet}{\mathsf{sSet}}
\newcommand{\bSet}{\mathsf{bSet}}
\newcommand{\TT}{\mathsf{tt}}
\newcommand{\FF}{\mathbb{F}}
\newcommand{\Iso}{{\sf Iso}}
\newcommand{\refl}{{\bf refl}}
\newcommand{\HH}{{\sf H}}


\newcommand{\mkbox}[1]{\ensuremath{#1}}


\newcommand{\Id}{{\sf Id}}
\newcommand{\ident}{{\sf id}}
\newcommand{\Path}{{\sf Path}}
\newcommand{\IdP}{{\sf IdP}}
\newcommand{\ID}{{\sf ID}}
\newcommand{\Equiv}{{\sf Equiv}}
\newcommand{\isEquiv}{{\sf isEquiv}}
\newcommand{\ext}{{\sf ext}}
\newcommand{\isContr}{{\sf isContr}}
\newcommand{\Fib}{\mathsf{Fib}}
\newcommand{\Susp}{\mathsf{Susp}}
\newcommand{\susp}{\mathsf{susp}}
\newcommand{\Fill}{\mathsf{Fill}}

\newcommand{\CC}{{\mathcal C}}
\newcommand{\subst}{{\sf subst}}
\newcommand{\res}{{\sf res}}
\newcommand{\Int}{{\bf I}}
\newcommand{\sem}[1]{\langle #1\rangle}

\newcommand{\Sph}{{\sf S}^1}
\newcommand{\PROP}{{\sf prop}}
\newcommand{\SET}{{\sf set}}
\newcommand{\pair}[1]{{\langle #1 \rangle}}
\newcommand{\Prod}[2]{\displaystyle\prod _{#1}~#2}
\newcommand{\Sum}[2]{\displaystyle\sum _{#1}~#2}
\newcommand{\gothic}{\mathfrak}
\newcommand{\omicron}{*}
\newcommand{\gP}{{\gothic p}}
\newcommand{\lift}[1]{\tilde{#1}}
\newcommand{\gM}{{\gothic M}}
\newcommand{\gN}{{\gothic N}}
\newcommand{\rats}{\mathbb{Q}}
\newcommand{\ints}{\mathbb{Z}}

%\newtheorem{proposition}[theorem]{Proposition}

%\documentstyle{article}
\newcommand{\IF}[3]{{{\sf if}~#1~{\sf then}~#2~{\sf else}~#3}}
\newcommand{\lfpi}[3]{(\Pi #1{:}#2)#3}
\newcommand{\HA}{{\sf HA}}
\newcommand{\AC}{{\sf AC}}
\newcommand{\HAw}{\hbox{\sf{HA}$^{\omega}$}}
\newcommand{\EM}{\hbox{\sf{EM}}}
\newcommand{\DC}{\hbox{\sf{DC}}}
\newcommand{\BB}{\hbox{\sf{B}}}
\def\Box{\hbox{\sf b}}

\def\NN{\hbox{\sf N}}
\def\Type{\mathsf{Type}}
\def\KType{\mathsf{KType}}
\def\FType{\mathsf{FType}}
%\def\Box{\hbox{\sf B}}
\def\PER{\hbox{\sf PER}}
\def\FUN{\Pi}
\def\ELEM{\hbox{\sf El}}
\def\GG{\hbox{\sf G}}
\def\TP{\hbox{\sf TP}}
\def\N0{\hbox{\sf N}_0}
\def\ZERO{\hbox{\sf zero}}
\def\SUCC{\hbox{s}}
\setlength{\oddsidemargin}{0in} % so, left margin is 1in
\setlength{\textwidth}{6.27in} % so, right margin is 1in
\setlength{\topmargin}{0in} % so, top margin is 1in
\setlength{\headheight}{0in}
\setlength{\headsep}{0in}
\setlength{\textheight}{9.19in} % so, foot margin is 1.5in
\setlength{\footskip}{.8in}

% Definition of \placetitle
% Want to do an alternative which takes arguments
% for the names, authors etc.

\newcommand{\lbr}{\lbrack\!\lbrack}
\newcommand{\rbr}{\rbrack\!\rbrack}
%\newcommand{\sem}[2] {\lbr #1 \rbr_{#2}}  % interpretation of the terms
\newcommand{\PAIR}[2] {{<}#1,#2{>}}  % interpretation of the terms
\newcommand{\add}{\mathsf{add}}
\newcommand{\app}{\mathsf{app}}
\newcommand{\APP}{\mathsf{APP}}
\newcommand{\BAPP}[2]{\mathsf{app}(#1,#2)}
\newcommand{\nat}{{N}}
\newcommand{\NNO}{\hbox{\sf N$_0$}}
\newcommand{\UU}{\hbox{\sf U}}
\newcommand{\VV}{\hbox{\sf V}}
\newcommand{\EXIT}{\mathsf{exit}}
\newcommand{\natrec}{\hbox{\sf{natrec}}}
\newcommand{\boolrec}{\hbox{\sf{boolrec}}}
\newcommand{\nil}{[]}
\newcommand{\cons}{\mathsf{cons}}
\newcommand{\lists}{\mathsf{list}}
\newcommand{\VEC}{\mathsf{vec}}
\newcommand{\reclist}{\mathsf{RecL}}
\newcommand{\vect}{\mathsf{vect}}
\newcommand{\brecp}{\Psi}
\newcommand{\true}{\mathsf{true}}
\newcommand{\false}{\mathsf{false}}
\newcommand{\bool}{{N_2}}
\newcommand{\ifte}[3]{\mathsf{if}\ #1\ \mathsf{then}\ #2\ \mathsf{else}\ #3}
\newcommand{\nats}{\mathbb{N}}
\newcommand{\Con}{{\sf Con}}
\newcommand{\Typ}{{\sf Type}}
\newcommand{\Elem}{{\sf Elem}}
\newcommand{\Char}{{\sf Char}}
%\newcommand{\id}{{\sf id}}
\newcommand{\id}{{1}}
\newcommand{\mm}{{\sf m}}
\newcommand{\qq}{{\sf q}}
\newcommand{\COMP}[3]{{\sf comp}~#1~#2~#3}
\newcommand{\comp}{{\sf comp}}
\newcommand{\hcomp}{{\sf hcomp}}
\newcommand{\genComp}{{\sf Comp}}
\newcommand{\pres}{{\sf pres}}
\newcommand{\extend}{{\sf extend}}
\newcommand{\eq}{{\sf equiv}}

\newcommand{\Transp}{{\sf fill}}
\newcommand{\Glue}{{\sf Glue}}
\newcommand{\glue}{{\sf glue}}
\newcommand{\Comp}{{\sf fill}}
% Marc's macros
\newcommand{\op}[1]{#1^\mathit{op}}
\newcommand{\set}[1]{\{#1\}} 
\newcommand{\es}{\emptyset}
\newcommand{\lto}{\longmapsto}
\newcommand{\rup}[1]{#1{\uparrow}}
\newcommand{\rdo}[1]{#1{\downarrow}}
\newcommand{\rupx}[1]{#1{\uparrow_{x}}}
\newcommand{\rdox}[1]{#1{\downarrow_{x}}}
\newcommand{\rupxy}[1]{#1{\uparrow_{x,y}}}
\newcommand{\rdoxy}[1]{#1{\downarrow_{x,y}}}
\newcommand{\rupyx}[1]{#1{\uparrow_{y,x}}}
\newcommand{\rdoyx}[1]{#1{\downarrow_{y,x}}}
\newcommand{\del}[1]{}
\newcommand{\ul}[1]{\underline{#1}}
\newcommand{\bind}[2]{{\langle}#1{\rangle}#2}
\newcommand{\lam}[2]{{\langle}#1{\rangle}#2}
\newcommand{\make}[1]{{\langle}#1{\rangle}}
\newcommand{\OO}{O}

% end Marc's macros
\begin{document}

\title{Simplicial set semantics of higher inductive types}

\author{}
\date{}
\maketitle

%\rightfooter{}

\section*{Introduction}

 The goal of this note is to show that, by combining some results in
the references\cite{CHM}, \cite{GS} and \cite{OP}, we do get a model of higher inductive
types in simplicial sets, extending and simplifying Voevodsky's model of
univalent type theory. This follows essentially what was announced by Andrew Swan in 
some 
discussions\footnote{This discussion can be found at
https://groups.google.com/d/msg/homotopytypetheory/bNHRnGiF5R4/3RYz1YFmBQAJ.}
and this note tries to record the details of this argument.

 To simplify the presentation, we limit ourselves to explain how to represent
the suspension operation as an operation in $U\rightarrow U$ where $U$ is some
univalent universe in the simplicial set model.

 Another goal of this note is to record the fact that the ideas used for
the cubical set model, presented axiomatically in \cite{OP}, can also be
combined with \cite{Sattler} to give a quite simple way to build the
standard Quillen model structure on {\em simplicial} sets. The use of classical
logic is limited to one point (for establishing the logical equivalence of the
``uniform'' notion of Kan fibration with the usual definition).

\section{A reformulation of Kan composition}

 Like in \cite{OP}, we use the internal language of presheaf models.
We write $I,J,K,\dots$ the objects of the base category (nonempty
finite linear posets).
We write $\FF$ the presheaf such that $\FF(I)$ is the set of decidable
sieves on $I$. (If the metalanguage is classical then $\FF = \Omega$
is the subobject classifier, but we want here to limit the use of classical
logic to one place.) We let $\II$ be the presehaf $\Delta^1$.
Internally, $\II$ has a (bounded) distributive lattice structure
and the can follow the setting of \cite{CHM,OP}.
We have a dependent type $[\psi]$ for $\psi:\FF$ where $[\psi]$
is given by $[\psi](I,S) = \{0~|~1_I\in S\}$.
A {\em filling operation} for a dependent type $A$ over a type $\Gamma$
is an operation which given $\gamma:\Gamma^{\II}$ and $\psi:\FF$ and a partial section
in $\Pi (i:\II) [\psi\vee i = b]\rightarrow A\gamma(i)$ with $b = 0$ or $1$,
extends it to a total section in $\Pi (i:\II) A\gamma(i)$.
We write $\Fill(\Gamma,A)$ the type of such operations.

\medskip

 If $\Gamma$ is the terminal object we get the notion of {\em fibrancy} structure.
A fibrancy structure on a presheaf $X$ is an operation which, given
$\psi:\FF$ and a partial section in $\Pi (i:\II)[\psi\vee i = b]\rightarrow X$
extends it to a total section in $X^{\II}$.
We write $\Fib(X)$ the type of such operations.

 In general to give only $\Pi(\rho:\Gamma)\Fib(A\rho)$ is weaker than 
to give an element in $\Fib(\Gamma,A)$.

 In \cite{CHM} we present a type of {\em transport structure}, which 
together with an element in  $\Pi(\rho:\Gamma)\Fib(A\rho)$ produces
an element in  $\Fib(\Gamma,A)$.
It is given by an operation which give $\psi$ and $\gamma:\Gamma^{\II}$
which is {\em constant} on $\psi$ and a partial section in
$\Pi (i:\II)[\psi\vee i = b]\rightarrow A\gamma(i)$ which is {\em constant}
on $\psi$, extends it to a total section in $\Pi (i:\II)A\gamma(i)$.

\section{Suspension operation}

 Given $X$, we define a $\Susp~X$-algebra to be a type $A$ with a fibrancy
structure $h_A$ and two points $n_A,s_A$ and a family of paths $l_A:X\rightarrow A^{\II}$
connecting $n_A$ to $s_A$. There is a natural notion of $\Susp~X$-algebra and
we can show \cite{CHM} externally\footnote{Thanks to one referee for pointing out to us
that we don't need any special property of the interval for this operation, and thus that it
works as well for {\em simplicial} sets.} that for any $X$ there exists an inital $\Susp~X$-algebra
denoted simply by $\Susp~X$. It has three constructors $\north,~\south:\Susp~X$
and $\merid~x~i:\Susp~X$ for $x:X$ and $i:\II$.

\begin{theorem}
  $\susp{X}$ satisfies the dependent elimination rule for the
  suspension: given a family of type $P$ over $\susp{X}$ with a
  composition structure, and $n$ in $P~\north$ and $s$ in $P~\south$
  and $l~x~i$ in $P~(\merid~x~i)$ such that $l~x~0 = n$ and $l~x~1 =            
  s$ there exists a map $\elim:\Pi (x:\susp{X})P~x$ such that
  $\elim~\north = n$ and $\elim~\south = s$ and $\elim~(\merid~x~i) =           
  l~x~i$.
\end{theorem}

 If $A$ is a dependent type over $\Gamma$ then we define $\Susp~A$ by
$(\Susp~A)\rho = \Susp~(A\rho)$. It is then possible to show
\cite{CHM} that from any element in $\Fill(\Gamma,A)$ we can build
an element in $\Fill(\Gamma,\Susp~A)$.

  Let ${\cal U}$ be a Grothendieck universe. If $A$ is ${\cal U}$-valued
presheaf on the category of elements of $\Gamma$, then so is
$\Susp~A$.

\section{Kan fibration and simplicial set model}

 We can relate this internal notion of filling structure to the usual
notion of Kan fibration.

\begin{theorem}\label{main}
If $\Gamma$ is a presheaf and $A$ a presheaf on the category of elements of 
$\Gamma$ then the following conditions are equivalent
\begin{enumerate}
\item $\Gamma.A\rightarrow\Gamma$ is a Kan fibration
\item $\Gamma.A\rightarrow\Gamma$ has the right lifting property w.r.t. any pushout product
of a monomorphism and an endpoint inclusion in $\Delta^1$
\item there exists an element in $\Fill(\Gamma,A)$.
\end{enumerate}
\end{theorem}

\begin{proof}
The equivalence between the two first points is a classic result in the theory
of simplicial sets (e.g. Goerss-Jardine, Proposition 4.2; this is the only place
where one uses classical logic, and more precisely decidability of degeneracy and 
axiom of choice). The equivalence
of the second and third conditions is proved elegantly in \cite{GS}, by using the
notion of Leibnitz product and exponential.
\end{proof}

 We introduce the following notation $\Type_0(\Gamma)$ is the set
of ${\cal U}$-valued presheaves on the category of elements of $\Gamma$,
and  $\FType_0(\Gamma)$ is the set
of ${\cal U}$-valued presheaves on the category of elements of $\Gamma$
{\em together with} a filling structure
and  $\KType_0(\Gamma)$ is the subpresheaf of $\Type_0(\Gamma)$
of ${\cal U}$-valued presheaves on the category of elements of $\Gamma$
for which {\em there exists} a filling structure. By Theorem \ref{main},
$\KType_0(\Gamma)$ is equivalently the set of ${\cal U}$-valued presheaves $A$ on the presheaf $\Gamma$
such that $\Gamma.A\rightarrow \Gamma$ is a Kan fibration.

All of these define 
presheaves on the category of presheaves in a canonical way.

 We define $U(I)$ to be the set $\KType_0(Yon(I))$.
We define a presheaf $El$ on the category of elements of $U$ by
$El(I,X) = X(I,1_I)$, so that $El$ is ${\cal U}$-valued. 
If $A$ is a ${\cal U}$-valued presheaf on the category
of elements of $\Gamma$ with a filling structure, there exists a unique map $|A|:\Gamma\rightarrow U$
such that $El|A| = A$.

 It can be shown that $U.El\rightarrow U$ is a Kan fibration.
By Theorem \ref{main}, we have a global element in $\Fill(U,El)$
(this is the only place where classical logic is used).

 We expand the difference between this model (where filling is a property)
and the ``cubical'' set models (where filling is a structure). If we
define $V(I)$ to be the set $\FType(Yon(I))$, then this defines a presheaf, but there
will not be a natural bijection between $\Gamma\rightarrow V$
and $\FType_0(\Gamma)$.
On the contrary, when we define $U(I)$ as above, then there is a natural
bijection between the set $\Gamma\rightarrow U$
and $\KType_0(\Gamma)$.

We have a map $\susp:U\rightarrow U$ such that 
$El (\susp~X) = \Susp~(El X)$ for $X:U$.

 To show that we have an elimination rule, we proceed as for showing the existence
of the elimination rule for the identity type in the simplicial set model.
We consider the context (using extension types notation)
$$
X:U,~P:El(\susp~X)\rightarrow U,~n:P~\north,~s:P~\south,
~l:\Pi (x:El X)\Pi (i:\II)P~(\merid~x~i)[i=0\mapsto n,~i=1\mapsto s]
$$
and in this context (like in \cite{CHM}) we 
build an element in $\elim:\Pi (x:El (\susp~X))P~x$ such that
$\elim~\north = n$ and $\elim~\south = s$ and $\elim~(\merid~x~i) =           
l~x~i$.

\section{Quillen model structure on simplicial sets and one conjecture}

 Using \cite{OP}, it can be shown that $U$ is fibrant
(i.e. has a fibrant structure). We can follow \cite{Sattler}
and build a model structure on simplicial sets. The only use of
classical logic is in the existence of a filling structure in $\Fill(U,El)$
which is a consequence of Theorem \ref{main}.

\medskip

 It follows also from what we presented that it is possible to interpret in simplicial
sets the version of cubical type theory (based on distributive lattice) where 
the composition operation are new {\em constants} (they satisfy only the substitution laws
but no computation rules for composition of dependent products and sums and paths and
universes). This system has also models in cubical sets (where we can compute).
A conjecture is that it should be possible to use the glueing technique
(as in \cite{Shulman}) to show that this formal system satisfies Voevodsky's conjecture: any
closed term of type natural numbers should be path equal to a numeral.
This would be one way to show that various versions of cubical type theory (that are extensions
of this ``constant'' version by new computation rules) give the same values for a closed
term of type natural numbers in ordinary dependent type theory extended with univalence.

\begin{thebibliography}{9}

\bibitem{CHM}
Th. Coquand, S. Huber and A. M\"ortberg.
\newblock{On Higher Inductive Types in Cubical Type Theory.}
\newblock{LICS 2018.}

\bibitem{GS}
N. Gambino and Ch. Sattler.
\newblock{The Frobenius condition, right properness, and uniform fibrations.}
\newblock{Journal of Pure and Applied Algebra, 221 (12), 2017, pp. 3027-3068.}

\bibitem{OP}
I. Orton and A. Pitts.
\newblock{Axioms for modelling cubical type theory in a topos.}
\newblock{CSL 2016.}


\bibitem{Sattler}
Ch. Sattler.
\newblock{The equivalence extension property and model structures.}
\newblock{Submitted, 2017, https://arxiv.org/pdf/1704.06911.}

\bibitem{Shulman}
M. Shulman.
\newblock{Univalence for inverse diagrams and homotopy canonicity.}
\newblock{MSCS, p. 1203-1277, 2015.}

\end{thebibliography}

\end{document}      
                                                                                  
 
